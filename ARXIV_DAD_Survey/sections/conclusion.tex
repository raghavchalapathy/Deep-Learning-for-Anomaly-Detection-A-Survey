%!TEX root = ../main.tex
\section{Conclusion}
\label{sec:chapter1_conclusion}
In this chapter we have discussed various research methods in deep learning-based anomaly detection
alongwith its application across various domains. This article discusses the challenges in deep anomaly detection and presents  several existing solutions to these challenges. For each category of deep anomaly detection techniques, we present the assumption regarding the notion of normal and anomalous data along with its strength and weakness.  The goal of this survey was to investigate and identify the various deep learning models  for anomaly detection and evaluate its suitability for a given dataset. When choosing a deep learning model to a particular domain or data, these assumptions can be used as guidelines to assess the effectiveness of the technique in that domain. Deep learning based anomaly detection is still an active research, a possible future work would be to extend and update as more mature techniques are proposed.



